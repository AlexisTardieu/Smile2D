\documentclass[11pt]{beamer}
\usefonttheme{default}
%\usetheme{Montpellier}
\usetheme[progressbar=frametitle]{metropolis}

\usepackage[utf8]{inputenc}
\usepackage[french]{babel}
\usepackage{pgf,tikz,pgfplots}
%\pgfplotsset{compat=1.15}
\usepackage{mathrsfs}
\usetikzlibrary{arrows}

\usepackage[utf8]{inputenc}
\usepackage[french]{babel}
\usepackage[T1]{fontenc}
\usepackage{amsmath}
\usepackage{amsfonts}
\usepackage{amssymb}
%\usepackage[dvipsnames]{xcolor}
\usepackage{enumitem}
\frenchbsetup{StandardLists=true} 

\usepackage{caption}
\usepackage{subcaption}



\def \bs {\backslash}
\def \eps {\varepsilon}
\def \p {\varphi}

\def \un {\mathds{1}}
\def \RR {\mathbb{R}}
\def \EE {\mathbb{E}}
\def \CC {\mathbb{C}}
\def \KK {\mathbb{K}}
\def \NN {\mathbb{N}}
\def \PP {\mathbb{P}}
\def \TT {\mathbb{T}}
\def \ZZ {\mathbb{Z}}
\def \sA {\mathcal{A}}
\def \sC {\mathcal{C}}
\def \sF {\mathcal{F}}
\def \sM {\mathcal{M}}
\def \sO {\mathcal{O}}
\def \sG {\mathcal{G}}
\def \sE {\mathcal{E}}
\def \sB {\mathcal{B}}
\def \sS {\mathcal{S}}
\def \sT {\mathcal{T}}
\def \sD {\mathcal{D}}
\def \sH {\mathcal{H}}
\def \sN {\mathcal{N}}
\def \sP {\mathcal{P}}
\def \sQ {\mathcal{Q}}
\def \sU {\mathcal{U}}
\def \sV {\mathcal{V}}
\def \sL {\mathcal{L}}
\def \disp {\displaystyle}
\def \vn {\vspace{3 mm} \noindent}
\def \ni {\noindent}
\def \red {\textcolor{Red}}
\def \blue {\textcolor{Blue}}
\def \green {\textcolor{OliveGreen}}
\def \brown {\textcolor{RawSienna}}
\def \purple {\textcolor{Purple}}
\def \navy {\textcolor{NavyBlue}}
\def \orange {\textcolor{Orange}}
\def \pink {\textcolor{Magenta}}
\def \d {{\rm d}}
\def \div {\text{div}}
\def \rot {\nabla \times}
\def \grad {\nabla}
\def \lap {\Delta}
\def \vit {\vec{u}}
\def \grav {\vec{g}}
\newcommand{\peq}{\mathrel{+}=}
\newcommand{\meq}{\mathrel{-}=}
\newcommand{\feq}{\mathrel{*}=}

\def \HRule {\rule{\linewidth}{0.5mm}}





\newcommand{\themename}{\textbf{\textsc{metropolis}}\xspace}
\metroset{block=fill}
\setbeamertemplate{frame numbering}[fraction]
%\titlegraphic{\hfill\includegraphics[height=1.3cm]{index.png}}
\renewcommand{\thesubsection}{\alph{subsection}}

\author{\Large{Valentin Pannetier $-$ Alexis Tardieu}}
\title{
		\Large{\hspace{5mm} Résolution des équations de l'élasticité} \\
		\Large{\hspace{10mm} linéaire sur grille cartésienne 2D $-$} \\
		\Large{\hspace{20mm} Application au morphing.}
}
%\setbeamercovered{transparent} 
\setbeamertemplate{navigation symbols}{} 
%\logo{} 
\institute{M2 \textit{Modélisation Numérique \& Calcul Haute Performance} - Univeristé de Bordeaux}
\date{Vendredi 29 Janvier 2020} 
%\subject{} 



\begin{document}



\begin{frame}
\maketitle
\end{frame}



\begin{frame}{Plan de l'exposé}
\setbeamertemplate{section in toc}[sections numbered]
\setbeamertemplate{subsection in toc}[subsections numbered]
\tableofcontents[hideallsubsections]
\end{frame}




\section{Introduction}



\begin{frame}{Introduction}



\end{frame}



\begin{frame}{Introduction}
	
	
	
\end{frame}



\section{Les équations de l'élasticité linéaire en 2D}



\begin{frame}{Les équations de l'élasticité linéaire en 2D}

$\bullet$~~ \'Equations de l'élasticité linéaire sur $\Omega \subset \RR^2$ :
$$- \div (\sigma (\vec{u})) = \vec{f}$$

où l'inconnue est le déplacement $\vec{u} = [u,v]^T \in \RR^2$.

\vn
$\bullet$~~ Le tenseur des contraintes :
$$\sigma (\vec{u}) = \begin{bmatrix}
\sigma_{1,1} & \sigma_{1,2} \\
\sigma_{2,1} & \sigma_{2,2}
\end{bmatrix}$$

est donné par la loi suivante :
$$[\sigma (\vec{u})]_{i,j} = 2 \mu [\eps (\vec{u})]_{i,j} + \lambda \div (\vec{u}) \delta_{i,j}$$


\end{frame}



\begin{frame}{Les équations de l'élasticité linéaire en 2D}
	
$\bullet$~~ Tenseur des déformations :
$$\eps(\vec{u}) := \frac{1}{2} [\grad \vec{u} + \grad^T \vec{u}] = \frac{1}{2} \begin{bmatrix}
\disp 2 \frac{\partial u}{\partial x} & \disp \frac{\partial u}{\partial y} + \frac{\partial v}{\partial x} \\[2mm]
\disp \frac{\partial u}{\partial y} + \frac{\partial v}{\partial x} & \disp 2 \frac{\partial v}{\partial y}
\end{bmatrix}$$

\vn
$\Rightarrow$~~ Expression du tenseur des contraintes :
$$\sigma (\vec{u}) = \begin{bmatrix}
\disp (2 \mu + \lambda) \frac{\partial u}{\partial x} + \lambda \frac{\partial v}{\partial y} & \disp \mu \left( \frac{\partial u}{\partial y} + \frac{\partial v}{\partial x} \right) \\[2mm]
\disp \mu \left( \frac{\partial u}{\partial y} + \frac{\partial v}{\partial x} \right) & \disp (2 \mu + \lambda) \frac{\partial v}{\partial y} + \lambda \frac{\partial u}{\partial x}
\end{bmatrix}$$
	
\end{frame}



\begin{frame}{Les équations de l'élasticité linéaire en 2D}
	
$\bullet$~~ Obtention d'un système d'EDP couplées, où $\vec{f} = [f_1, f_2]^T$ :
$$\begin{array}{ll}
\disp - \left[ \frac{\partial \sigma_{1,1}}{\partial x} + \frac{\partial \sigma_{1,2}}{\partial y} \right]  = f_1 \\[3mm]
\disp - \left[ \frac{\partial \sigma_{2,1}}{\partial x} + \frac{\partial \sigma_{2,2}}{\partial y} \right]  = f_2
\end{array}$$

\vn
soit en extension :
$$\begin{array}{ll}
\disp \left[ (2 \mu + \lambda) \cdot \frac{\partial^2}{\partial x^2} + \mu \cdot \frac{\partial^2}{\partial y^2} \right] (u) + \left[ (\lambda + \mu) \cdot \frac{\partial^2}{\partial x \partial y} \right] (v) = - f_1 \\[3mm]
\disp \left[ (\lambda + \mu) \cdot \frac{\partial^2}{\partial x \partial y} \right] (u) + \left[ (2 \mu + \lambda) \cdot \frac{\partial^2}{\partial y^2} + \mu \cdot \frac{\partial^2}{\partial x^2} \right] (v) = - f_2
\end{array}$$
	
\end{frame}



\begin{frame}{Les équations de l'élasticité linéaire en 2D}
	
$\bullet$~~ Mise sous forme matricielle :
$$\begin{bmatrix}
\disp (2 \mu + \lambda) \cdot \frac{\partial^2}{\partial x^2} + \mu \cdot \frac{\partial^2}{\partial y^2} & \disp (\lambda + \mu) \cdot \frac{\partial^2}{\partial x \partial y} \\
\disp (\lambda + \mu) \cdot \frac{\partial^2}{\partial x \partial y} & \disp (2 \mu + \lambda) \cdot \frac{\partial^2}{\partial y^2} + \mu \cdot \frac{\partial^2}{\partial x^2}
\end{bmatrix} \begin{bmatrix}
u \\ v
\end{bmatrix} = \begin{bmatrix}
- f_1 \\ - f_2
\end{bmatrix}$$

\vn
$\bullet$~~ Formellement : système linéaire à résoudre !

\vn
$\Rightarrow$~~ 3 opérateurs à discrétiser en différences finies :
$$L_x \equiv \frac{\partial^2}{\partial x^2} \hspace{15mm} L_y \equiv \frac{\partial^2}{\partial y^2} \hspace{15mm} H \equiv \frac{\partial^2}{\partial x \partial y}$$
	
\end{frame}



\section{La méthode des différences finies}



\begin{frame}{La méthode des différences finies}
	
Schéma grille cartésienne : $Nx, Ny, \Delta x, \Delta y$ + un stencil complet (9 pts)
	
\end{frame}



\begin{frame}{La méthode des différences finies}
	
Vecteur d'inconnues + explication tailles $L_x, L_y, H$
	
\end{frame}



\begin{frame}{La méthode des différences finies}
	
$\bullet$~~ L'opérateur $L_x \equiv \partial_x^2$ :
$$L_x (w_{i,j}) = \frac{w_{i+1,j} - 2 w_{i,j} + w_{i-1,j}}{\Delta x^2}$$

\vn
soit sous forme matricielle :

{\small $$L_x = \frac{1}{\Delta x^2} \left[ \begin{array}{ccc|ccc|ccc}
-2 & 1 &  &       &  &  &      &  &  \\
1 & -2 & 1 &      &  &  &     &  &  \\
& 1 & -2 &        &  &  &       &  &  \\

\hline

&  &  &      -2 & 1 &  &       &  &  \\
&  &  &      1 & -2 & 1 &      &  &  \\
&  &  &       & 1 & -2 &       &  &  \\

\hline

&  &  &       &  &  &      -2 & 1 &  \\
&  &  &       &  &  &      1 & -2 & 1 \\
&  &  &       &  &  &       & 1 & -2 
\end{array} \right]$$}
	
\end{frame}



\begin{frame}{La méthode des différences finies}
	
$\bullet$~~ L'opérateur $L_y \equiv \partial_y^2$ :
$$L_y (w_{i,j}) = \frac{w_{i,j+1} - 2 w_{i,j} + w_{i,j-1}}{\Delta y^2}$$

\vn
soit sous forme matricielle :

{\small $$L_y = \frac{1}{\Delta y^2} \left[ \begin{array}{ccc|ccc|ccc}
	-2 &  &  &      1 &  &  &      &  &  \\
	 & -2 &  &      & 1 &  &     &  &  \\
	&  & -2 &        &  & 1 &       &  &  \\
	
	\hline
	
	1 &  &  &      -2 &  &  &      1 &  &  \\
	& 1 &  &       & -2 &  &      & 1 &  \\
	&  & 1 &       &  & -2 &       &  & 1 \\
	
	\hline
	
	&  &  &       1 &  &  &      -2 &  &  \\
	&  &  &       & 1 &  &       & -2 &  \\
	&  &  &       &  & 1 &       &  & -2 
\end{array} \right]$$}
	
\end{frame}



\begin{frame}{La méthode des différences finies}
	
$\bullet$~~ L'opérateur $H \equiv \partial_{x,y}^2$ :
$$H (w_{i,j}) = \frac{w_{i+1,j+1} - w_{i-1,j+1} - w_{i+1,j-1} + w_{i-1,j-1}}{4 \Delta x \Delta y}$$

\vn
soit sous forme matricielle :

{\small $$H = \frac{1}{4 \Delta x \Delta y} \left[ \begin{array}{ccc|ccc|ccc}
    &  &  &      & 1 &  &      &  &  \\
	&  &  &      -1 &  & 1 &      &  &  \\
	&  &  &      & -1  &  &      &  &  \\
	
	\hline
	
    & -1 &  &       &  &  &       & 1 &  \\
	1 &  &-1  &       &  &  &       -1 &  & 1 \\
	& 1 &  &       &  &  &       & -1 &  \\
	
	\hline
	
	&  &  &       & -1 &  &       &  &  \\
	&  &  &       1 &  & -1 &       &  &  \\
	&  &  &       & 1 &  &       &  &  
\end{array} \right]$$}
	
\end{frame}



\begin{frame}{La méthode des différences finies}
	
$\bullet$~~ Mise sous forme matricielle $AX = B$ :
$$\underbrace{\begin{bmatrix}
(2 \mu + \lambda) \cdot L_x + \mu \cdot L_y & (\lambda + \mu) \cdot H \\
(\lambda + \mu) \cdot H & (2 \mu + \lambda) \cdot L_y + \mu \cdot L_x
\end{bmatrix}}_{\disp = A} \underbrace{\begin{bmatrix}
U \\ V
\end{bmatrix}}_{\disp = X} = \underbrace{\begin{bmatrix}
- F_1 \\ - F_2
\end{bmatrix}}_{\disp = B}$$

\vn
$\bullet$~~ Système de taille $N = 2 \cdot Nx \cdot Ny$... mais essentiellement creux $\Rightarrow$ utilisation de la bibliothèque \textbf{Eigen} pour les \textit{SparseMatrix}

\vn
$\bullet$~~ Résolution : gradient conjugué d'Eigen (en $\mathcal{O}(N)$)
	
\end{frame}



\begin{frame}{La méthode des différences finies}
	
	
	
\end{frame}



\begin{frame}{La méthode des différences finies}
	
	
	
\end{frame}


\section{Un point sur les méthodes level-set}



\begin{frame}{Un point sur les méthodes level-set}
	
\textbf{Problème :} résolution d'EDP sur $\Omega \subset \RR^2$ non cartésien $\Rightarrow$ la frontière $\Gamma = \partial \Omega$ ne coïncide pas avec des points de la grille.

\begin{center}
	\includegraphics[width=9cm]{part4-img1.png}
\end{center}

$\Rightarrow$ Comment distinguer efficacement les points intérieurs à $\Omega$ ?
	
\end{frame}



\begin{frame}{Un point sur les méthodes level-set}

$\bullet$~~ Point de départ : domaine $\Omega \subset \RR^2$ de bord $\Gamma$ quelconque.

\begin{center}
	\includegraphics[width=6cm]{part4-img4.png}
\end{center}

$\bullet$~~ Fonction \textit{level-set} = fonction $\phi : \RR^2 \to \RR$ régulière telle que :
$$\phi(P) = \left| \begin{array}{l}
+ d (P, \Gamma) ~~\text{si } P \in \Omega^+ \\[2mm]
- d (P, \Gamma) ~~\text{si } P \in \Omega^-
\end{array} \right.$$

\end{frame}



\begin{frame}{Un point sur les méthodes level-set}
	
$\Rightarrow$ $\phi$ est une fonction \textit{distance algébrique}. Caractérisation évidente :
$$\left| \begin{array}{l}
\phi (P) > 0 ~~\text{si } P \in \Omega^+ \\[2mm]
\phi (P) = 0 ~~\text{si } P \in \Gamma \\[2mm]
\phi (P) < 0 ~~\text{si } P \in \Omega^-
\end{array} \right.$$

\vn
$\bullet$~~ 2 façons de définir une level-set $\phi$ :

\begin{itemize}[label=$-$]
	\item Forme \textit{implicite} : $\phi(x,y) = \cdots$ (expression déjà connue)
	
	\vn
	
	\item Forme \textit{paramétrique} : $x (\theta) = \cdots$ et $y (\theta) = \cdots$ (on ne connaît pas $\phi$ a priori, ou on ne peut pas l'expliciter)
\end{itemize}
	
\end{frame}



\begin{frame}{Un point sur les méthodes level-set}
	
$\bullet$~~ Cas implicite : cercle centré en $(0,0)$ et de rayon $R=1$ :
$$\phi (x,y) = \sqrt{x^2 + y^2} - R$$

\begin{center}
	\includegraphics[width=6cm]{part4-img2.png}
\end{center}

... ou bien $x(\theta) = R \cos(\theta)$ et $y(\theta) = R \sin(\theta)$ ?
	
\end{frame}



\begin{frame}{Un point sur les méthodes level-set}
	
	$\bullet$~~ Cas paramétrique : étoile de mer à 5 branches :
	$$x(\theta) = [R + k \sin(5 \theta)] \cos(\theta) \hspace{5mm} \text{et} \hspace{5mm} y(\theta) = [R + k \sin(5 \theta)] \sin(\theta)$$
	
	\begin{center}
		\includegraphics[width=6cm]{part4-img3.png}
	\end{center}
	
	... ou bien $\phi (x,y) = \sqrt{x^2 + y^2} - R [1 + k \sin(5 \theta)]$ ?
	
\end{frame}


\section{Application au morphing}



\begin{frame}{Application au morphing}
	
	
	
\end{frame}



\begin{frame}{Application au morphing}
	
	
	
\end{frame}



\begin{frame}{Application au morphing}
	
	
	
\end{frame}



\begin{frame}{Application au morphing}
	
	
	
\end{frame}


\section{Conclusion}



\begin{frame}{Conclusion}
	
	
	
\end{frame}










\end{document}